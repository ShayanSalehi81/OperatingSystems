قسمت‌های الکترونیکی در یک سامانه معمولا با توزیع‌های نمایی مدل‌سازی می‌شوند. دلیل اصلی این کار خاصیت بدون حافظه بودن این توزیع احتمالاتی است. در قطعات الکترونیکی بر خلاف قسمت‌های مکانیکی، از آنجایی که دچار استهلاک نیستند باید خاصیت
less Memory 
داشته باشند. در واقع احتمال هرلحظه نسبت به یک زمان قبلی ثابت باید یکسان باشد.

این قطعات معمولا با پارامترهایی مانند فرکانس، جریان،‌ ولتاژ و بقیه موارد در ارتباط است. همچنینی برای نقص این قسمت‌ها می‌توان از توزیع برنولی نیز استفاده کرد.

در آن طرف قسمت‌های مکانیکی را با توزیع نرمال یا وایبول معمولا مدل‌سازی می‌شوند. دلیل آن این است که به خاطر وجود عوامل خارجی همانند فرسایش و اصطکاک،‌ به مرور زمان توزیع احتمالاتی آن متفاوت است. در این قسمت‌های ویژگی‌های مواد مانند سختی، انعطاف پذیری و استحکام تاثیر دارد.

تفاوت‌های اعظم این دو قسمت، تفاوت در پارامترهای تاثیرگذار در این قسمت‌ها بوده و از این جهت توزیع‌های مختلف احتمالاتی در مدلسازی آنها استفاده می‌شود.

به طور کلی، مدل‌سازی قسمت‌های مکانیکی با استفاده از قوانین دینامیک، استاتیک، و مکانیک مواد انجام می‌شود، در حالی که مدل‌سازی قسمت‌های الکترونیکی با استفاده از قوانین الکترومغناطیس، الکترونیک، و دیگر اصول مرتبط با فیزیک الکترونیک انجام می‌پذیرد.

همچنین تجزیه و تحلیل قسمت‌های مکانیکی ممکن است نیاز به بررسی‌های مانند تحلیل تنش، تحلیل دینامیکی، یا تحلیل خستگی داشته باشد. در مقابل، تجزیه و تحلیل قسمت‌های الکترونیکی ممکن است به بررسی‌هایی مانند تحلیل فرکانس، تحلیل دایره‌های الکترونیکی یا تحلیل سیستم‌های دیجیتال نیاز داشته باشند.