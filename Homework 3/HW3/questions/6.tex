\subsection*{الف}
بله سیستم مورد نظر در حالت
\lr{safe}
قرار داشته زیرا که اگر تمام منابع
\lr{allocate}
شده را درنظر بگیریم برای منابع
$A$
،
$B$
و
$C$
خواهیم داشت:
\[
    Capacity(A) = 2, \quad Capacity(B) = 1, \quad Capacity(C) = 1
\]

برای اجرا پردازه‌ها نیز می‌توانیم روندی همانند روند زیر را داشته باشیم:
\[
    P_3 \to P_2 \to P_4 \to P_1    
\]
که روند اجرایی آنها با جزییات به صورت زیر خواهد بود:
\begin{align*}
    &Exec(P_3) \\
    &Capacity(A) = 1, \quad Capacity(B) = 1, \quad Capacity(C) = 1 \\
    &FreeResource(P_3) \\
    &Capacity(A) = 7, \quad Capacity(B) = 5, \quad Capacity(C) = 4 \\
    &Exec(P_2) \\
    &Capacity(A) = 3, \quad Capacity(B) = 4, \quad Capacity(C) = 4 \\
    &FreeResource(P_2) \\
    &Capacity(A) = 8, \quad Capacity(B) = 7, \quad Capacity(C) = 7 \\
    &Exec(P_4) \\
    &Capacity(A) = 6, \quad Capacity(B) = 0, \quad Capacity(C) = 7 \\
    &FreeResource(P_4) \\
    &Capacity(A) = 10, \quad Capacity(B) = 8, \quad Capacity(C) = 9 \\
    &Exec(P_1) \\
    &Capacity(A) = 8, \quad Capacity(B) = 0, \quad Capacity(C) = 8 \\
    &FreeResource(P_1) \\
    &Capacity(A) = 12, \quad Capacity(B) = 9, \quad Capacity(C) = 12 \\
\end{align*}
که بدون هیچ مشکلی تمام پردازه‌ها اجرا خواهند شد.

\subsection*{ب}
خیر، با اختصاص دو واحد از منبع
$A$
به پردازه
$P_1$
به حالت
\lr{Deadlock}
خواهیم رسید. زیراکه در آن لحظه
$Capacity(A) = 0$
شده و هیچ پردازه‌ای را نمی‌توانیم اجرایش را تمام کنیم که منابعش را آزاد کنیم، برای همین به بن‌بست خواهیم خورد.

\subsection*{ج}
در اینجا با توجه به دیاگرام اجرای پردازه‌ها در قسمت الف، از آنجایی که پردازه
$P_1$
آخرین پردازه‌ای است که اجرا می‌شود، به ظرفیت منابع در خط‌های پردازشی بالای آن نگاه می‌کنیم. از آنجایی که دو بار
$Capacity(B)$
به صفر می‌رسد. نمی‌توان از این منابع به این پردازه اختصاص داد در عوض از هر دو منابع
$A$
و
$C$
در همه حالات یک واحد اضافی داریم که آن را می‌توانیم به پردازه 
$P_1$
اختصاص دهیم. (در واقع مینیموم ظرفیت منابع در پردازه‌های قبلی حداکثر ظرفیتی است که در لحظه شروع می‌توان به پردازه اول اختصاص داد.)