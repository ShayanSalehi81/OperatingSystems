برای آنکه بتوانیم مطمئن شویم گراف داده شده از توالی پردازده‌ها به درستی رعایت می‌شوند می‌توانیم از دو سمافور
$S_1$
و
$S_2$
استفاده کنیم. به این صورت که از سمافور اولی برای ترتیب توالی پردازده‌
$T_1$
با
$T_2$
و
$T_3$
استفاده می‌کنیم و از سمافور دومی برای پردازه‌های منتی به
$T_4$
استفاده خواهیم کرد.

به این ترتیب داریم:

\LTR
\begin{verbatim}
    void T_1 () {
        T_1 Running;
        Signal(S_1);
        Signal(S_1);
    }

    void T_2 () {
        Wait(S_1);
        T_2 Running;
        Signal(S_2);
    }

    void T_3 () {
        Wait(S_1);
        T_3 Running;
        Signal(S_2);
    }

    void T_3 () {
        Wait(S_2);
        T_4 Running;
    }
\end{verbatim}
\RTL

در اینجا برای مقدار اولیه
$S_1$
و
$S_2$
داریم:
\[
    Value(S_1) = 0, \quad \quad Value(S_2) = -1    
\]
دلیل اینکار برای این است که پرادازه
$T_4$
باید منتظر بماند که هردو پردازه 
$T_3$
و
$T_2$
کار خود را به اتمام برسانند و تابع
\verb|Signal(S_2)|
را صدا بزنند.

به صورت کلی دو پردازه دوم و سوم در
\verb|Wait()|
$S_1$
خواهند ماند تا زمانی که پردازه اول کار خود را به اتمام برسانند و دوباره
\verb|Signal(S_1)|
را صدا بزند تا مقدار آن برابر دو شود و پردازه‌های بعدی به اجرا در بیایند.