\subsection*{الف}
نادرست، لزومی ندارد که به حالت
\lr{Deadlock}
بر نخوریم زیرا این حالت برسر منابع رخ داده همچنین علت‌های دیگری همانند
\lr{IO}
نیز می‌تواند باعث ایجاد بن‌بست در سیستم ما شود و اختصاص یک پردازنده به پردازه تضمینی برای عدم رخ دادن این مورد به ما نمی‌دهد.

\subsection*{ب}
نادرست، ممکن است پردازه‌های حذف شده تاثیر در ایجاد
\lr{Deadlock}
نداشته باشند و می‌بایست پردازه‌هایی که در حالت چرخه وجود داشته را شناسایی کرده و سپس حذف کردن آنها یکی از راه‌های ممکن برای حل این مشکل خواهد بود.

\subsection*{ج}
درست، در حین اجرای برنامه به 
\lr{Deadlock}
بر نخواهیم خورد به خاطر اینکه هر پردازنده تمام ریسورس‌های لازم خود را در ابتدا برداشته است و نیاز ندارد که در حین اجرا به ریسورس دیگر درخواست دهد. البته این در حالتی امکان‌ پذیر است که در همان ابتدا نیز بتوانیم این منابع را بین همه پردازه‌ها اختصاص دهیم و ممکن است در ابتدا این کار در سیستم ممکن نباشد.

\subsection*{د}
نادرست، لزوما این کار باعث از بین بردن
\lr{Deadlock}
نمی‌شود. اولویت بندی اختصاص منابع به پردازه‌ها می‌تواند باعث جلوگیری رخ دادن یک بن‌بست شود اما ممکن است در حالتی قرار گرفته باشیم که با هر اولویت بندی ممکن هم نتوانیم سیستم را از
\lr{Deadlock}
خارج کنیم. بنابراین این کار همیشه جوابگو نخواهد بود.