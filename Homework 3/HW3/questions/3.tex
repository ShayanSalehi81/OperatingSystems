الگوریتم 
\lr{Banker}
یکی از روش‌های پیشگیری از 
\lr{Deadlock}
در سیستم‌های عامل است. این الگوریتم از طریق مدیریت منابع سیستم به گونه‌ای عمل می‌کند که اطمینان حاصل شود سیستم همیشه در حالت
\lr{Safe State}
باقی می‌ماند. حالت امن به وضعیتی از سیستم گفته می‌شود که در آن، برای هر تقاضای منابع توسط فرایندها، این تضمین وجود دارد که منابع مورد نیاز بتوانند در زمانی معین و بدون ایجاد 
\lr{Deadlock}
تأمین شوند.

در الگوریتم 
\lr{Banker}
، هر فرایندی که به سیستم اضافه می‌شود، باید حداکثر تعداد منابع مورد نیاز خود را از پیش اعلام کند. الگوریتم سپس بررسی می‌کند که آیا پذیرش تقاضای فعلی فرایند و اختصاص منابع به آن، سیستم را به حالت غیر امن خواهد کشاند یا خیر. اگر پذیرش تقاضا منجر به حالت غیر امن شود، تقاضا معلق می‌ماند تا زمانی که تأمین منابع بدون ریسک بن‌بست امکان‌پذیر باشد.

به طور خلاصه، الگوریتم 
\lr{Banker}
بررسی می‌کند که آیا اختصاص منابع درخواستی به فرایند خاصی باعث ایجاد یک زنجیره بن‌بست می‌شود یا خیر. این کار با مقایسه درخواست‌های فرایندها با منابع موجود و تخمین اینکه آیا فرایندها می‌توانند به صورت ترتیبی تکمیل و منابع آزاد شوند یا خیر، انجام می‌گیرد. در صورتی که تمام فرایندها بتوانند بدون ایجاد بن‌بست کامل شوند، سیستم در حالت امن قرار دارد و درخواست منابع تأیید می‌شود. در غیر این صورت، درخواست‌ها تا زمان تغییر شرایط منابع به تعویق می‌افتند.

این رویکرد تضمین می‌کند که سیستم همواره در حالتی قرار دارد که بتواند به درخواست‌های منابع پاسخ دهد و در عین حال از بن‌بست جلوگیری کند، به طوری که هیچ یک از فرایندها به طور دائمی در انتظار منابع قرار نگیرند و سیستم بتواند به صورت موثر به فعالیت خود ادامه دهد.












