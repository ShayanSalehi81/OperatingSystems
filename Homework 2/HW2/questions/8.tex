\subsection*{آ}
این کد پس از ایجاد یک فرزند، هم در پردازه والد و هم در پردازه فرزند، فایل مورد نظر را باز کرده و در آن محتوایی را می‌نویسد. نکته قابل توجه این است که این دو پردازه ممکن است با سرعت‌های متفاوتی عمل کنند. اگر پردازه والد زودتر نوشته و سپس فرزند نوشتار خود را انجام دهد، نتیجه در فایل "a" خواهد بود. در صورتی که فرزند زودتر نویسندگی کند، نتیجه "b" خواهد بود، زیرا پردازه‌ای که دیرتر اقدام به نوشتن می‌کند، محتوای نوشته شده توسط پردازه قبلی را بازنویسی می‌کند. این اتفاق به دلیل اینکه فایل‌ها به طور جداگانه باز شده‌اند رخ می‌دهد و نوشتار یک پردازه تأثیری بر مکان‌نما \lr{(pointer)} در توصیف‌گر فایل \lr{(file description)} پردازه دیگر ندارد. بنابراین، پاسخ نهایی می‌تواند "a" یا "b" باشد.

\subsection*{ب}
در این شرایط، نکته کلیدی این است که بر خلاف آنچه شاید انتظار برود، فقط یک حرف در نهایت در فایل چاپ خواهد شد. دلیل این موضوع این است که توصیف‌گرهای فایل \lr{file descriptors} یا به اختصار \lr{fd} بین این دو پردازه مشترک نیستند. هر پردازه دارای توصیف‌گر فایل منحصر به فرد خود است و هر یک به صورت مستقل به نوشتن \lr{(write)} در فایل می‌پردازند.

بنابراین، اگر یکی از پردازه‌ها، چه پدر یا فرزند، دیرتر نسبت به دیگری اقدام به نوشتن در فایل کند، محتوای نوشته شده توسط پردازه قبلی توسط این پردازه بازنویسی خواهد شد. این بازنویسی به این معنی است که تنها آخرین نوشته (از پردازه‌ای که دیرتر نوشته است) در فایل باقی می‌ماند.

در نتیجه، اگر پردازه والد دیرتر نسبت به فرزند نوشته خود را انجام دهد، محتوای نوشته شده توسط فرزند بازنویسی شده و در نهایت "a" در فایل قرار خواهد گرفت. در صورتی که فرزند دیرتر اقدام به نوشتن کند، محتوای نوشته شده توسط والد جایگزین شده و نتیجه "b" خواهد بود. این بدان معناست که در هر دو حالت، امکان دارد که یکی از حروف "a" یا "b" به تنهایی در فایل باقی بماند.

\subsection*{ج}
در این بخش از کد، در تضاد با بخش‌های قبلی، توصیف‌گر فایل \lr{(fd)} به عنوان یک متغیر گلوبال تعریف شده و این امکان را فراهم می‌کند که بین یک پردازه و تردی که در آن ایجاد می‌شود به اشتراک گذاشته شود. در این کد، سه سناریو مختلف ممکن است رخ دهد:

\begin{itemize}
    \item تردی که تازه ایجاد شده ممکن است قبل از اجرای خط ۹ کد اصلی فعال شود.
    \item ممکن است پس از اجرای خط ۹ و قبل از اتمام کامل اجرای کد، ترد اجرا شود.
    \item همچنین ممکن است ترد فرصت اجرا پیدا نکند تا قبل از اینکه برنامه به پایان برسد.
\end{itemize}

در این سه حالت، نتایج مختلفی در فایل حاصل می‌شوند: در حالت اول، مقادیر "ab" در فایل نوشته خواهند شد. در حالت دوم، مقادیر "ba" در فایل ظاهر می‌شوند. و در حالت سوم، تنها مقدار "b" در فایل باقی خواهد ماند. این سه سناریو نشان‌دهنده این واقعیت هستند که ترتیب اجرای تردها و پردازه‌ها می‌تواند تأثیر قابل توجهی بر نتیجه نهایی داشته باشد.

\subsection*{د}
در حالت پیش‌فرض، با اتمام اجرای کد اصلی، تمام تردهایی که توسط آن کد ایجاد شده‌اند، خاتمه می‌یابند \lr{terminate} می‌شوند. اما با افزودن \verb|exit_pthread| در انتهای کد اصلی، امکان ادامه کار برای سایر تردها فراهم می‌شود و آن‌ها ترمینت نخواهند شد. این بدان معناست که در این حالت، برخلاف سناریویی که در آن ترد فرصت اجرا پیدا نمی‌کند، چنین موردی در کد رخ نمی‌دهد. بنابراین، در این حالت، هر دو نتیجه "ab" و "ba" می‌توانند در فایل نوشته شوند، چرا که هر دو ترد فرصت کافی برای اجرا قبل از پایان برنامه را خواهند داشت.

\subsection*{ه}
در این قسمت، اگر هنگام تغییر بافر و نوشتن در بافر به فایل توسط یک پردازه یا ترد، در بخش موازی دیگر کد هیچ تغییری در بافر ایجاد نشود و چیزی در فایل نوشته نشود، رفتار کد کاملاً مشابه با حالت قبلی خواهد بود. در نتیجه، همه خروجی‌های احتمالی موجود در حالت قبلی در این حالت نیز ممکن هستند. اما، اگر یکی از بخش‌های کد مقداری را در بافر قرار دهد و قبل از نوشتن آن در فایل، بخش دیگری از کد بافر را دوباره تغییر دهد، احتمال دارد که یک حرف به طور دوبار در فایل چاپ شود. بنابراین، علاوه بر احتمال وجود مقادیر "ab" و "ba" در فایل، احتمال وجود مقادیر "aa" و "bb" نیز وجود دارد.