\subsection*{آ}
اتفاقاتی که باید ذخیره شوند:

مقادیر رجیستر‌ها: شامل رجیسترهای محلی ریسه، مانند مقادیر پایینتر و بالاتر \lr{stack pointer}، \lr{program counter} و سایر رجیسترهای مربوط به وضعیت فعلی ریسه.

مقادیر مربوط به اجرای ریسه: این شامل مواردی مانند اولویت ریسه و وضعیت آماده به اجرا یا در انتظار است.

از آنجایی که هر دو ریسه به یک پردازه تعلق دارند، داده‌های مربوط به فضای کاربر پردازه (مانند فضای آدرس حافظه) مشترک هستند و نیازی به ذخیره و بازیابی مجدد آن‌ها در یک \lr{context switch} داخلی نیست.

\subsection*{ب}
مقادیر رجیستر‌ها: مشابه حالت الف، مقادیر رجیستر‌های هر ریسه باید ذخیره و بازیابی شوند.

فضای آدرس حافظه: از آنجایی که ریسه‌ها به پردازه‌های مختلف تعلق دارند، فضای آدرس حافظه هر پردازه (شامل بخش‌هایی مانند داده‌ها، کد و پشته) باید ذخیره و هنگام بازگشت به آن پردازه بازیابی شود.

اطلاعات مربوط به حافظه نهان و سایر منابع سیستم: این شامل اطلاعاتی است که برای حفظ وضعیت پردازه و بهینه‌سازی دسترسی‌های حافظه لازم است.
