\subsection*{آ}
در اینجا نرخ سرویس دهی برابر است با تعداد درخواست‌هایی که سرور می‌تواند در واحد زمانی سرویس دهد. با توجه به اینکه زمان سرویس دهی هر درخواست ۲۰ میلی ثانیه است، سرور می‌تواند در هر ثانیه ۵۰ درخواست را پردازش کند ($\frac{1}{0.02} = 50$).

با استفاده از این دو نرخ، می‌توانیم میزان استفاده از سرور ($\rho$) را محاسبه کنیم، که برابر است با نسبت نرخ رسیدن به نرخ سرویس دهی ($\rho = \frac{\lambda }{\mu}$). در این حالت، $\rho$ برابر است با $1.2 = \frac{60}{50}$. از آنجایی که $\rho$ بیشتر از ۱ است، این نشان دهنده این است که سرور قادر به پردازش تمام درخواست‌های ورودی در زمان واقعی نیست و صف درخواست‌ها به مرور زمان افزایش می‌یابد. در این حالت، تاخیر صف به بینهایت میل می‌کند.

\subsection*{ب}
در اینجا فاصله بین درخواست‌ها ۳۰ میلی ثانیه بوده که یعنی در هر ثانیه 
$\frac{1000}{30} = 33$
درخواست داریم.

در این حالت، نسبت استفاده از سرور ($\rho$) برابر است با $\frac{33}{50} = 0.66$. این نشان می‌دهد که سرور قادر به پردازش تمام درخواست‌های ورودی است و صف در نهایت به حالت پایدار می‌رسد. تاخیر صف در حالت پایدار به عواملی مانند تعداد اولیه درخواست‌ها در صف و نرخ سرویس دهی بستگی دارد، اما با توجه به اینکه $\rho$ کمتر از ۱ است، صف به طور مداوم از بین نخواهد رفت و در نهایت به یک تعادل خواهد رسید.


\subsection*{ج}
میزان بهره‌وری سرور در زمان طولانی به نسبت استفاده از سرور ($\rho$) بستگی دارد. در این مثال، از آنجایی که $\rho$ برابر با $0.66$ است، می‌توان گفت که سرور در حدود $66.6\%$ زمان خود را صرف پردازش درخواست‌ها می‌کند و $33.3\%$ زمان در حالت آماده‌به‌کار بدون پردازش درخواست است. این نشان می‌دهد که سرور به طور موثری استفاده می‌شود و فضای کافی برای پردازش درخواست‌های اضافی وجود دارد.