در اینجا ابتدا در ترمینال 
\lr{Starting main}
چاپ شده و پس از آن به خاطر وجود خط

\LTR
\verb|dup2(file_fd, STDOUT_FILENO);|
\RTL

مابقی خروجی‌ها در این فایل نوشته خواهند شد.

در خط ۵ متود
\verb|fork()|
صدا زده شده و در پروسه والد در شرط
\verb|if|
رفته و در حالت 
\verb|wait|
می‌مانیم.

اما در پردازه فرزند درون
\verb|else|
رفته و مقدار درون آن در فایل ذخیره می‌گردد و همچنین به دلیل آنکه
\verb|child_pid|
در پردازه فرزند صفر است خروجی صفر خواهیم داشت.

پس از اتمام کار پردازه فرزند، وارد پردازه والد شده و مقدار
۶۶۶۶
به عنوان
\verb|child_pid|
شناخته و به این ترتیب در فایل خروجی به چنین حالتی برخواهیم خورد.

\LTR
\begin{center}
    \verb|In child|

    \verb|Ending main: 0|

    \verb|In parent|

    \verb|Ending main: 6666|
\end{center}
\RTL