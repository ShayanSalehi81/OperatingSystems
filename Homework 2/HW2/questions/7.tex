در این کد، ابتدا عدد ۱ چاپ می‌شود، سپس ۵ ترد ایجاد می‌کنیم که قرار است به ترتیب اعداد ۲ تا ۶ را چاپ کنند. پس از آن، منتظر تکمیل اجرای ۴ ترد اول می‌مانیم و در نهایت عدد ۷ چاپ می‌شود. برای محاسبه کل حالات ممکن، به تحلیل سناریوها می‌پردازیم.

اگر عدد ۶ توسط ترد آخر به موقع چاپ نشود و قبل از پایان برنامه ظاهر نگردد، فقط اعداد چاپ شده توسط ۴ ترد اول به هر ترتیبی ممکن است نمایان شوند و در نهایت عدد ۷ چاپ می‌شود. بنابراین، در این حالت !۴ حالت وجود دارد.

در صورتی که عدد ۶ نیز در خروجی چاپ شده باشد، ۴ عدد اول !۴ حالت مختلف دارند. همچنین، خود عدد ۶ می‌تواند قبل از اولین عدد چاپ شده در تردهای دیگر یا حتی بعد از عدد ۷ چاپ شود، پس خود آن ۶ حالت مختلف دارد. در نتیجه، در کل ۶ × !۴ حالت وجود دارد.

به این ترتیب برای تمامی رشته‌های متمایز خواهیم داشت:

\[
    6 \times 4! + 4! = 7 \times 4!  
\]