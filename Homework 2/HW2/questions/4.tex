\textbf{زمانبندی Round-Robin بدون در نظر گرفتن اولویت‌ها:}

در این حالت، تمام پردازه‌ها بدون توجه به اولویت‌شان و بر اساس زمان ورود و کوانتوم زمانی ۶ میلی‌ثانیه‌ای، اجرا می‌شوند. این شامل تغییرات زیر است:

\begin{itemize}
    \item هر پردازه به مدت ۶ میلی‌ثانیه اجرا شده و سپس جای خود را به پردازه بعدی می‌دهد.

    \item در صورتی که پردازه‌ای \lr{io-bound} باشد مانند D، پس از ۱ میلی‌ثانیه به صف \lr{io} منتقل می‌شود و پس از ۲ میلی‌ثانیه به صف \lr{ready} بازمی‌گردد.

    \item تغییرات و وضعیت هر پردازه در هر زمان در جدول ثبت می‌شود.
\end{itemize}

\textbf{زمانبندی Round-Robin با در نظر گرفتن Preemption و اولویت وظایف در لحظه ورود:}

در این حالت، وظایف بر اساس اولویت‌شان و با در نظر گرفتن \lr{preemption} اجرا می‌شوند:

\begin{itemize}
    \item هر وظیفه بر اساس اولویت و زمان ورود خود در صف \lr{ready} قرار می‌گیرد.

    \item پردازه با اولویت بالاتر در هر لحظه جای خود را به پردازه با اولویت پایین‌تر می‌دهد.

    \item مانند حالت قبل، پردازه‌های \lr{io-bound} به صف io منتقل می‌شوند و پس از آن به صف \lr{ready} بازمی‌گردند.
\end{itemize}

\textbf{زمانبندی FCFS با در نظر گرفتن اولویت و بدون Preemption:}

در این حالت، وظایف بر اساس زمان ورود و اولویت اجرا می‌شوند، اما بدون \lr{preemption}:

\begin{itemize}
    \item وظایف به ترتیب زمان ورود و اولویت در صف \lr{ready} قرار می‌گیرند.

    \item هر پردازه به طور کامل اجرا می‌شود قبل از اینکه پردازه بعدی شروع به کار کند.
    
    \item پردازه‌های \lr{io-bound} به همان شکل قبل به صف \lr{io} و سپس به صف \lr{ready} منتقل می‌شوند.
\end{itemize}

برای محاسبه \lr{turnaround time} هر وظیفه، زمان پایان هر وظیفه منهای زمان ورود آن محاسبه می‌شود. سپس، میانگین این زمان‌ها برای کل وظایف به دست می‌آید. این محاسبات نیاز به توجه دقیق به جزئیات و دنبال کردن هر پردازه در طول زمان دارد.