\subsection*{الف}
در اینجا از آنجایی که میانگین زمان دسترسی به حافظه 
$50ns$
بوده و دسترسی
\lr{TLB}
هم
$10ns$
است. حال دو حالت داریم:
\begin{itemize}
    \item
    در حالت
    \lr{TLB hit}
    زمان دسترسی به این صورت است:
    \[
        time\,to\,access\,memory = 10\,ns(TLB) + 50\,ns(Memory) = 60\,ns    
    \]

    \item
    در حالت
    \lr{TLB miss}
    داریم:
    \[
        time\,to\,access\,memory = 10\,ns(TLB) + 50\,ns(Page Table) + 50\,ns(Memory) = 110\,ns    
    \]
\end{itemize}
حال با توجه به ضریب
\lr{TLB}
۵۰ درصد خواهیم داشت:
\[
    Avarage\,Time = \frac{1}{2} \times 60\,ns + \frac{1}{2} \times 110\,ns = 85\,ns    
\]

حال درصورتی که
\lr{TLB}
نداشته باشیم داریم:
\[
    Avarage\,Time = 50\,ns(Page Table) + 50\,ns(Memory) = 100\,ns   
\]

با قیاس کردن این دو حالت می‌توان فهمید که روش
\lr{TLB}
با ضریب ۵۰ درصد می‌تواند میانگین زمان دسترسی را ۱۵ نانو ثانیه کاهش دهد.

\subsection*{ب}
اگر ضریب
\lr{hit rate}
را
$H$
در نظر گرفته و زمان میانگین را
$T$
در نظر بگیریم داریم:
\[
    T = H \times (Time\,for\,TLB\,hit)  + (1 - H) \times (Time\,for\,TLB\,miss)
\]
پس داریم:
\[
    T = 61\,ns, \quad Time\,for\,TLB\,hit = 60\,ns, \quad Time\,for\,TLB\,miss = 110\,ns
\]
\[
    \implies 61 = H \times 60 + (1 - H) \times 110 \implies 50 H = 49 \implies H = 0.98
\]
بنابراین ضریب
\lr{TLB}
می‌بایست ۹۸ درصد باشد تا میانگین زمان ۶۱ نانو ثانیه شود..