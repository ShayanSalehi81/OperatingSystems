در اینجا می‌دانیم که به صورت کلی در سیستم‌های ۳۲ بیتی
\lr{PTE size}
۴ بایت یا همان ۳۲ بیت است.
حال برای پیدا کردن تعداد
\lr{PTE}ها
خواهیم داشت:
\[
    Number\,of\,PTEs = \frac{Page\,Size}{PTE\,Size} = \frac{4096}{4} = 1024
\]
در مورد استفاده از حافظه واقعی برای کمتر کردن سایز
\lr{Page Table}،
سیستم عامل بخشی از حافظه فیزیکی (واقعی) رو برای ذخیره جدول‌های صفحه استفاده می‌کنه. اما، لازم نیست همه جدول‌های صفحه کاملاً پر شده باشند یا حتی همیشه در حافظه باشند.

همچنین روش
\lr{Demand Paging}
سیستم عامل اجازه می‌ده فقط اون صفحاتی (و در نتیجه جدول‌های صفحه) رو به حافظه بارگذاری کنه که در حال حاضر نیازه، و اینکار اثر حافظه‌ای جدول‌های صفحه رو کاهش می‌ده.

تو سیستم جدول صفحه چند سطحی، فقط جدول صفحه سطح بالاتر (سطح-۲ در این مورد) باید کاملاً در حافظه حضور داشته باشه. جدول‌های صفحه سطح پایین‌تر (سطح-۱) به موقع بارگذاری می‌شن. این کار به شدت میزان حافظه واقعی لازم برای جدول‌های صفحه رو کاهش می‌ده.

در مورد آدرس ۳۲ بیتی هم ما سه بخش زیر را داریم:
\begin{itemize}
    \item
    آفست ۱۲ بیتی: مکان دقیق درون یک صفحه رو مشخص می‌کنه (چون$2^{12} = 4096$، که اندازه صفحه است).
    آفست ۱۲ بیتی برای پیدا کردن بایت دقیق در صفحه ۴ کیلوبایتی که فرآیند می‌خواد بهش دسترسی داشته باشه، استفاده می‌شه.
    \item
    جدول صفحه سطح-۱ ۱۰ بیتی: این قسمت آدرس برای ایندکس کردن در جدول صفحه سطح-۱ استفاده می‌شه.
    ۱۰ بیت برای جدول صفحه سطح-۱ استفاده می‌شه تا در جدول صفحه سطح-۱ مشخص شده (که توسط \lr{PTE} سطح-۲ اشاره شده) ایندکس بشه، که دوباره می‌تونه ۱۰۲۴ ورودی داشته باشه.
    \item
    جدول صفحه سطح-۲ ۱۰ بیتی: این قسمت آدرس برای ایندکس کردن در جدول صفحه سطح-۲ استفاده می‌شه.
    ۱۰ بیت برای جدول صفحه سطح-۲، یکی از ۱۰۲۴ ($2^{10}$) ورودی‌های موجود در جدول صفحه سطح-۲ رو ایندکس می‌کنه. این ورودی به یک جدول صفحه سطح-۱ در حافظه اشاره می‌کنه.
\end{itemize}

این ساختار اجازه می‌ده تا مدیریت یک فضای آدرس مجازی بزرگ (۴ گیگابایت در یک سیستم ۳۲ بیتی) انجام بشه، در حالی که اثر حافظه‌ای جدول‌های صفحه رو با استفاده از تکنیک‌های صفحه‌بندی سلسله مراتبی و صفحه‌بندی تقاضامحور، قابل مدیریت نگه می‌داره.