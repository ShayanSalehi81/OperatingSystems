در جواب به این سوال می‌توان گفت که پیج‌های حافظه به یک اندازه نیستند و به یک اندازه بودن آنها می‌تواند مشکلاتی همانند مشکلات زیر به وجود بیارن:
\begin{itemize}
    \item بهره‌وری ناکارآمد حافظه (تکه‌تکه شدن داخلی): وقتی فرآیندها به کل اندازه استاندارد صفحه نیاز ندارن، بخش استفاده نشده صفحه هدر می‌ره. به این موضوع تکه‌تکه شدن داخلی می‌گن.
    \item مشکلات عملکرد برای داده‌های حجیم: برای فرآیندهایی که با مجموعه‌های داده بزرگ سروکار دارن، استفاده از صفحات با اندازه استاندارد می‌تونه منجر به افزایش بار مدیریت صفحات شه و دسترسی به حافظه رو کند کنه.
    \item انعطاف‌پذیری محدود: صفحات با اندازه ثابت ممکنه برای همه نوع برنامه‌ها، به خصوص اونهایی که نیازهای متغیر حافظه‌ای دارن، بهینه نباشن.
\end{itemize}
حالا راه‌هایی که می‌تونیم این مشکل رو حل کنیم به این صورته:
\begin{itemize}
    \item اندازه‌های متفاوت صفحه (ترکیبی از اندازه‌های صفحه): بعضی از سیستم‌های مدرن از بیشتر از یک اندازه صفحه پشتیبانی می‌کنن. این روش به سیستم عامل اجازه می‌ده تا حافظه رو به شکلی تخصیص بده که بیشتر به اندازه مجموعه داده‌های یک فرآیند مناسب باشه. مثلاً، صفحات بزرگ می‌تونن برای برنامه‌هایی که نیاز به حافظه زیادی دارن استفاده شن، در حالی که صفحات کوچکتر برای فرآیندهای کم‌تقاضا به کار می‌رن.
    \item صفحات بزرگ و صفحات عظیم (\lr{HugePages}): تو لینوکس و بعضی از سیستم‌های عامل شبه یونیکس، یه ویژگی به نام \lr{HugePages} وجود داره که اجازه استفاده از اندازه‌های بزرگتر صفحه (مثل ۲ مگابایت یا ۱ گیگابایت) رو برای برنامه‌های خاص می‌ده، که باعث کاهش بار مدیریت تعداد زیادی صفحات می‌شه.
    \item صفحه‌بندی تقاضامحور و جابجایی (\lr{Swapping}): اگرچه راه‌حل مستقیمی برای مشکل اندازه صفحه نیست، صفحه‌بندی تقاضامحور (بارگذاری صفحات به حافظه فقط زمانی که نیاز هستند) و جابجایی (انتقال صفحات بین رم و دیسک) می‌تونه تاثیر اندازه‌های ناکارآمد صفحه رو با بهینه‌سازی نحوه استفاده از حافظه کاهش بده.
\end{itemize}
هر کدوم از این راه‌حل‌ها با معایب خودشون همراه هستن. مثلاً، در حالی که اندازه‌های بزرگتر صفحه می‌تونه بار مدیریت تعداد زیادی صفحه رو کم کنه، می‌تونه منجر به تکه‌تکه شدن داخلی بیشتر شه اگر درست مدیریت نشه. به همین دلیل، سیستم‌های عامل اغلب مکانیزم‌هایی رو فراهم می‌کنن تا این عوامل رو بر اساس نیازهای سیستم و برنامه‌هایش تعادل ببخشن.