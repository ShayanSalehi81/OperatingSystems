\subsection*{الف}
از آنجایی که هر
\lr{PTE}
۸ بیت برای آدرس فیزیکی داشته می‌دانیم که با این بیت‌ها
$2^8 = 256$
آدرس مختلف را می‌توان پوشش داد. همچنین اندازه هر صفحه براساس بیت‌های آفست تعیین شده که براساس معماری داده شده ۴ است. بنابراین هر صفحه می‌توان
$2^4 = 16$
بایت را آدرس‌دهی کند.

پس در نهایت بیشترین سایز آدرس فیزیکی برابر خواهد بود با:
\[
  256 \times 16 = 4096  
\]

\subsection*{ب}
انداره فضای آدرس‌دهی مجازی به تعداد بیت‌های
\lr{virtual address}
مربوط می‌شود که در این حالت خاص ۸ بیت بوده پس ماکزیموم فضای آن می‌تواند
$2^8 = 256$
باشد.

\subsection*{ج}
از آنجایی که تعداد بیت‌های آفست ۴ بیت بوده اندازه هر صفحه برابر با‍
$2^4 = 16$
بایت خواهد بود.

\subsection*{د}
ابتدا آدرس داده شده را از مبنای ۱۶ به مبنای دو می‌بریم:
\[
    (0x63) \implies (01100011)_2
\]
از آنجایی که با ارزش‌ترین رقم صفر بوده پس این آدرس از طریق
\lr{MMU}
آدرس‌دهی نمی‌شود. بنابراین ۷ بیت باقی‌مانده همان آدرس فیزیکی را نشان می‌هند که برابر ۶۳ در مبنای ۱۶ است.

\subsection*{ه}
\lr{translations}
های ولید آنهایی هستند که در
\lr{page table}
بیت ولید ۱ داشته باشند. آدرس‌های زیر شامل این حالت می‌شوند:
\[
    0xFF, \quad 0x08, \quad 0xB0, \quad 0xEE, \quad 0xDD 
\]
بنابراین رنج آدرس‌های ولید متناسب با آنها به این صورت خواهد بود:
\begin{latin}
    \begin{verbatim}
        0xFF00   to    0xFF0F
        0x0800   to    0x080F
        0xB000   to    0xB00F
        0xEE00   to    0xEE0F
        0xDD00   to    0xDD0F
    \end{verbatim}
\end{latin}
