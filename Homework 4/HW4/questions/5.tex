\subsection*{الف}
برای آنکه کمترین تعداد جدول صفحات را پیدا کنیم می‌دانیم می‌دانیم که
\lr{Page Size}
ما ۶۴ کیلوبایت بوده و سایز
\lr{Entry}
ما ۴ بایت بوده، به این ترتیب داریم:

\[ 
    \text{Number of Entries} = \frac{\text{Page Size}}{\text{Page Table Entry Size}} = \frac{65536 \text{ bytes}}{4 \text{ bytes/entry}} = 16,384
\]
هر جدول صفحه می‌تونه ۱۶۳۸۴ ورودی داشته باشه (همانطور که محاسبه شد). چون هر جدول صفحه در یک صفحه جا می‌گیره، حداقل اندازه لازم برای هر جدول صفحه، اندازه یک صفحه است، که ۶۴ کیلوبایت است.

\subsection*{ب}
در ابتدا می‌بایست بیت‌های آفست را تعیین کنیم، به این ترتیب داریم:
\[ 
    \text{Offset Bits} = \log_2(65536) = 16 \text{ bits} 
\]
آز آنجایی که آدرس مجازی ما ۲۴ بیت بوده خواهیم داشت:
\[ 
    \text{Page Table Bits} = \text{Total Virtual Address Bits} - \text{Offset Bits} = 24 - 16 = 8 \text{ bits} 
\]
چون ۸ بیت برای جدول صفحه استفاده می‌شود، و هر ورودی مربوط به یک صفحه است، یک جدول صفحه سطح یک برای آدرس‌دهی کل فضای آدرس مجازی کافی است.

\subsection*{ج}
چون فضای آدرس فیزیکی ۳۲-بیتی است، ورودی‌های جدول صفحه باید شامل شماره صفحه فیزیکی به علاوه هر متادیتا دیگری باشند. با فرض اینکه کل آدرس فیزیکی ۳۲-بیتی برای آدرس‌دهی استفاده می‌شه (که شامل شماره صفحه فیزیکی و آفست درون صفحه می‌شه)، تعداد بیت‌های موجود برای متادیتا بستگی به این داره که چند بیت برای نمایش شماره صفحه فیزیکی لازمه.

شماره صفحه فیزیکی توسط فضای آدرس فیزیکی منهای بیت‌های آفست (که در این مورد برای صفحات فیزیکی و مجازی یکسان است) تعیین می‌شود:

\[ 
    \text{Physical Page Number Bits} = \text{Physical Address Space} - \text{Offset Bits} = 32 - 16 = 16 \text{ bits} 
\]
بنابراین، تعداد بیت‌های موجود برای متادیتا در هر ورودی:

\[ 
    \text{Metadata Bits} = \text{Page Table Entry Size} - \text{Physical Page Number Bits} = 32 - 16 = 16 \text{ bits} 
\]
پس، هر ورودی در جدول صفحه می‌تونه ۱۶ بیت متادیتا داشته باشد.