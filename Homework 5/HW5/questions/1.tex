\subsection*{الف}
در اینجا می‌دانیم که هر
\lr{direct index}
به فضای ۱۰۲۴ بایتی اشاره کرده و از طرفی آز آنجایی که هر پوینتر ۵ بایت جا می‌گیرد برای
\lr{indirect index}
می‌دانیم که به یک فضای 
$1024/4 = 256$
بایتی اشاره می‌کند. همچنین برای 
\lr{2-level indirect index}
می‌دانیم دو مرحله از این فضادهی را خواهیم داشت. به این ترتیب برای کل فضایی که می‌توان آدرس‌دهی کرد داریم:
\[
    \overbrace{64 \times 1024}^{\text{\lr{direct index}}} + \overbrace{256 \times 1024}^{\text{\lr{indirect index}}} + \overbrace{256 \times 256 \times 1024}^{\text{\lr{2-level indirect index}}}
\]
\[
    \implies (2^6 + 2^8 + 2^16) \times 2^10 \implies \text{کل فضایی که می‌توان آدرس دهی کرد}
\]

\subsection*{ب}
از آنجایی که فایل در موقعیت 
۳۰۰۰۳۱۲
قرار دارد و بلوک‌های ۱۰۲۴ تایی‌ داریم این موقیعت در خانه
\linebreak
$\lfloor 3000312 / 1024 \rfloor = 2930$
قرار دارد. بنا‌براین این قسمت مربوط به آدرس‌دهی
\lr{2-level indirect index}
می‌شود که نیاز به ۳ عمل خواندن از حافظه یا 
\lr{disk access}
دارد.