در اینجا می‌خواهیم به سه روش گفته شده بپردازیم:

\textit{روش \lr{Linked}:}

در این روش اتفاقی که می‌افتد آن است که هر بلاک به بلاک بعدی اشاره می‌کند و برای پیدا کردن یک آدرس از بلاک بعدی می‌بایست آدرس بلاک فعلی را بخوانیم. در اینجا می‌خواهیم به بلوک فایل دهم برسیم پس نیاز داریم که ۱۰ بار از دیسک عمل خواندن را داشته باشیم.

\textit{روش \lr{Contiguous}:}

در این روش تمامی بلاک‌ها پشت سرهم بوده و با دانستن اندازه هر بلاک و آدرس بلاک اول می‌توان به هر بلاک دسترسی پیدا کرد. پس در اینجا برای خواندن بلاک دهم تنها یک خواندن از دیسک نیاز داریم زیراکه با دانستن آدرس بلاک اول می‌توانیم مستقیما به آن دسترسی پیدا کنیم.

\textit{روش \lr{Indexed}:}

در این روش می‌دانیم که خود جدول ایندکس‌های در مکانی دیگر ذخیره شده و این امکان را به ما می‌دهد که با یکبار خواندن آن از حافظه آدرس تمام بلاک‌ها را پیدا کنیم و از روی آن بخوانیم. پس در این روش به دو بار خواندن نیاز داریم، یک بار برای خواندن جدول ایندکس‌ها و بار دیگر برای خواندن خود بلاک دهم.